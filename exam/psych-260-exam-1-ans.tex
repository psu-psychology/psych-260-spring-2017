\documentclass[answers]{exam}
\usepackage{graphicx}
\usepackage{wrapfig}
\usepackage[utf8]{inputenc}

\title{PSYCH 260 Exam 1}
\author{}
\date{February 8, 2017}

\pagestyle{headandfoot}
\firstpageheader{PSY 260}{Section 003}{Exam 1}
\runningheader{PSY 260}{Section 003}{Exam 1}
\firstpagefooter{}{Page \thepage\ of \numpages}{}
\runningfooter{}{Page \thepage\ of \numpages}{}

\begin{document}
\maketitle

\begin{center}
  \fbox{\fbox{\parbox{5.5in}{\centering
        Answer the questions using the Scantron form.}}}
\end{center}
\vspace{0.1in}
\makebox[\textwidth]{Name:\enspace\hrulefill}

\newpage

\section{Main}

\begin{questions}

\begin{figure}[h]
\includegraphics[width=0.9\textwidth]{img/forebrain-3.jpg}
\centering
\end{figure}

%1
\question Identify the structure
\begin{choices}
\correctchoice Frontal lobe
\choice Parietal lobe
\choice Occipital lobe
\choice Temporal lobe
\end{choices}

%2
\question Identify the structure
\begin{choices}
\choice Forebrain
\correctchoice Midbrain
\choice Hindbrain
\choice Spinal cord
\end{choices}

%3
\question Identify the structure
\begin{choices}
\choice 4th ventricle
\choice Medulla
\choice Cerebellum
\correctchoice Pons
\end{choices}

%4
\question Identify the structure
\begin{choices}
\choice 4th ventricle
\correctchoice Medulla
\choice Cerebellum
\choice Pons
\end{choices}

\newpage

\begin{figure}[h]
\includegraphics[width=0.9\textwidth]{img/forebrain-3.jpg}
\centering
\end{figure}

%5
\question Identify the structure
\begin{choices}
\choice 4th ventricle
\choice Medulla
\correctchoice Cerebellum
\choice Pons
\end{choices}

%6
\question Identify the structure
\begin{choices}
\choice Frontal lobe
\choice Parietal lobe
\correctchoice Occipital lobe
\choice Temporal lobe
\end{choices}

%7
\question Identify the structure
\begin{choices}
\choice Frontal lobe
\correctchoice Parietal lobe
\choice Occipital lobe
\choice Temporal lobe
\end{choices}

%8
\question These tissues provide external structural support and protection for the CNS.
\begin{choices}
\choice Astrocytes
\correctchoice Meninges
\choice Cerebral ventricles
\choice Circle of Willis
\end{choices}

\newpage

\begin{figure}[h]
\includegraphics[width=0.80\textwidth]{img/three-brains.jpg}
\centering
\end{figure}

%9
\question What plane of section is represented in the left panel?
\begin{choices}
\choice Coronal
\choice Sagittal
\correctchoice Axial/horizontal
\choice Dorsal
\end{choices}

%10
\question What plane of section is represented in the middle panel?
\begin{choices}
\choice Coronal
\correctchoice Sagittal
\choice Axial/horizontal
\choice Dorsal
\end{choices}

%11
\question What plane of section is represented in the right panel?
\begin{choices}
\correctchoice Coronal
\choice Sagittal
\choice Axial/horizontal
\choice Dorsal
\end{choices}

%12
\question What fissure or sulcus is represented in the figures?
\begin{choices}
\choice Superior temporal sulcus
\choice Central sulcus
\choice Longitudinal fissure
\correctchoice Lateral fissure
\end{choices}

%13
\question Primary somatosensory cortex (SI) is found in the \fillin.
\begin{choices}
\choice Temporal lobe
\choice Frontal lobe
\choice Hypothalamus
\choice Basal ganglia
\correctchoice Parietal lobe
\end{choices}

\newpage

%14
\question Which of the following statements about neurons is \emph{incorrect}?
\begin{choices}
\choice Neurons have very long lives.
\choice Neurons can extend over long distances.
\correctchoice Neurons are the only cells that have negative resting potentials.
\choice Neurons use both electrical and chemical mechanisms to communicate.
\end{choices}

%15
\question Primary motor cortex is found in the \fillin.
\begin{choices}
\choice Temporal lobe
\correctchoice Frontal lobe
\choice Hypothalamus
\choice Basal ganglia
\choice Parietal lobe
\end{choices}

%16
\question Your grandmother has a stroke. The neurologist chooses an X-ray-based structural brain imaging method that gives satisfactory, but not especially detailed spatial resolution. What method is that?
\begin{choices}
\correctchoice Computed tomography (CT).
\choice functional MRI.
\choice Positron Emission Tomography (PET).
\choice Anterograde tract tracers.
\end{choices}

%17
\question The caudate nucleus is part of the \fillin.
\begin{choices}
\choice Temporal lobe
\choice Frontal lobe
\choice Hypothalamus
\correctchoice Basal ganglia
\choice Parietal lobe
\end{choices}

%18
\question The \fillin plays a role in biologically crucial behaviors, including those associated with ingestion (eating and drinking) and reproduction.
\begin{choices}
\choice Temporal lobe
\choice Frontal lobe
\correctchoice Hypothalamus
\choice Basal ganglia
\choice Parietal lobe
\end{choices}

%19
\question The typical flow of information through neurons begins with input on the \fillin and ends with output from the \fillin.
\begin{choices}
\choice axon; dendrites.
\choice soma; dendrites.
\correctchoice dendrites; terminal button.
\choice terminal button; soma.
\end{choices}

%20
\question Among other functions \fillin play(s) a role in regulating the extracellular concentration of \fillin.
\begin{choices}
\correctchoice astrocytes; glutamate.
\choice myelin sheath; Na+ ions.
\choice Circle of Willis; blood loss.
\choice blood/brain barrier; oxygen levels.
\end{choices}

\newpage

%21
\question Scientists are exploring how chronic conditions like depression can change the size and shape of brain structures using high resolution whole brain imaging techniques like \fillin.
\begin{choices}
\choice electroencephalography (EEG).
\choice hemodynamic response imaging.
\correctchoice structural MRI.
\choice Computed Tomography (CT).
\end{choices}

%22
\question How many neurons are there in the human brain?
\begin{choices}
\correctchoice About 86 billion.
\choice About 86 million.
\choice About the same number of seconds as in the average lifetime.
\choice It can't be estimated.
\end{choices}

%23
\question This type of glial cell provides neurons in the peripheral nervous system (PNS) with a myelin sheath.
\begin{choices}
\correctchoice Schwann cells
\choice Oligodendrocytes
\choice Microglia
\choice Purkinje cells
\end{choices}

%24
\question The hippocampus plays a central role in \fillin.
\begin{choices}
\choice Sexual behavior
\choice Metabolic, physical support of neurons
\choice Sensory relay processing
\correctchoice Memory storage and retrieval
\choice CNS protection
\end{choices}

%25
\question The thalamus serves this function, among others.
\begin{choices}
\choice Metabolic, physical support of neurons
\correctchoice Sensory relay
\choice Preparation for action
\choice Memory storage and retrieval
\choice CNS protection
\end{choices}

%26
% \question The sympathetic nervous system is crucial for
% \begin{choices}
% \choice Sexual behavior
% \choice Metabolic, physical support of neurons
% \choice Sensory relay
% \correctchoice Preparation for action
% \choice Memory storage and retrieval
% \end{choices}

\question Sodium (Na+) is highly concentrated \fillin. This means that the force of diffusion acting alone will push Na+ \fillin.
\begin{choices}
\choice inside; inward
\correctchoice outside; inward
\choice inside; outward
\choice outside; outward
\end{choices}

\newpage

%27
\question You're having trouble sleeping, so your physician orders a sleep study using polysomnography. You spend a night in the hospital with electrodes on your scalp. This is an example use case of \fillin.
\begin{choices}
\correctchoice electroencephalograpy (EEG).
\choice Multi-unit recording.
\choice transcranial magnetic stimulation.
\choice optical imaging.
\end{choices}

%28
\question \fillin, a type of glial cell, help regulate local blood oxygen levels in response to neuronal activity. These cells thus contribute to the signal measured by \fillin.
\begin{choices}
\choice oligodendrocytes; MEG
\choice Schwann cells; structural MRI
\correctchoice astrocytes; functional MRI
\choice microglia; structural and functional MRI
\end{choices}

%29
\question The neurotransmitters dopamine, norepinephrine, and serotonin originate from nuclei clustered in which midbrain region?
\begin{choices}
\choice Basal ganglia
\choice Lateral geniculate nucleus
\correctchoice Tegmentum
\choice Medial frontal cortex
\end{choices}

%30
\question The hypothalamus is NOT responsible for which of the following functions?
\begin{choices}
\choice Fleeing
\choice Feeding
\choice Fighting
\correctchoice Falling
\end{choices}

%31
\question Which of the following marks the medial boundary of the frontal lobe?
\begin{choices}
\choice Lateral fissure
\correctchoice Longitudinal fissure
\choice Central sulcus
\choice Inferior temporal gyrus
\end{choices}

%32
\question This type of myelinating cell, found in the CNS, ensheaths many neurons at once.
\begin{choices}
\choice Astrocytes
\correctchoice Oligodendrocytes
\choice Microglia cells
\choice Stellate cells
\end{choices}

%33
% \question Nodes of Ranvier, or gaps in the myelination of an axon, serve which purpose?
% \begin{choices}
% \correctchoice Increase the speed of propagation
% \choice Allow space in the axon for neurotransmitter release
% \choice Provide structural support to the neuron
% \choice Combine input from different dendrites
% \end{choices}

\newpage

%33
\question Descartes thought that this midbrain structure was the place where the soul interacted with the body to create movement by inflating the muscles.
\begin{choices}
\choice Pons
\choice Cerebral aqueduct
\correctchoice pineal gland
\choice Superior colliclus
\end{choices}

%34
\question When a neuron is “at rest,” which of the following ions are more heavily concentrated \emph{outside} of the cell?
\begin{choices}
\correctchoice Na+ and Cl-
\choice K+ and A-
\choice Na+ and K+
\choice Cl- and A-
\end{choices}

%35
\question When a neuron's membrane potential reaches the threshold for an action potential, \fillin.
\begin{choices}
\choice voltage-gated K+ channels close
\choice voltage-gated Na+ channels close and inactivate
\choice the Na/K pump works even harder to keep the concentration balance.
\correctchoice voltage-gated Na+ channels open
\end{choices}

%36
\question This part of the cell functions as the neuron’s “antennae” by serving as the primary place for receiving input.
\begin{choices}
\choice Axon
\choice Soma
\correctchoice Dendrites
\choice Terminal Buttons
\end{choices}

%37
\question During the rising phase of the action potential, \fillin channels \fillin.
\begin{choices}
\choice Ligand-gated K+; close
\choice Voltage-gated Na+; close
\correctchoice Voltage-gated Na+; open
\choice Voltage-gated K+; close
\end{choices}

%38
\question Neurons ensheathed in myelin conduct action potentials \fillin than those without myelin.
\begin{choices}
\choice more slowly
\correctchoice more quickly
\choice more slowly and efficiently
\choice more quickly, but less efficiently
\end{choices}

%39
% \question During the \emph{absolute} refractory period, a neuron will \fillin.
% \begin{choices}
% \choice fire again in response to an especially strong input.
% \choice produce an action potential that is twice the normal size.
% \choice open voltage-gated Ca++ channels.
% \correctchoice not fire no matter the strength of the input.
% \end{choices}

%39
\question \fillin are a type of glial cell that contributes to the Blood Oxygen-Level Dependent (BOLD) response measured in \fillin brain imaging. 
\begin{choices}
\correctchoice Astrocytes; fMRI
\choice Schwann cells; structural MRI
\choice Oligodendrocytes; EEG
\choice Stellate cells; PET
\end{choices}

\newpage

%40
% \question When an action potential reaches the axon terminal, \fillin open and this causes synaptic vesicles to fuse with the presynaptic membrane and release neurotransmitter into \fillin.
% \begin{choices}
% \correctchoice Voltage-gated Ca++ channels; the synaptic cleft.
% \choice Ligand-gated Cl- channels; the Nodes of Ranvier.
% \choice Na+/K+ pumps; the soma.
% \choice Passive/leak channels; post-synaptic autoreceptors.
% \end{choices}

%40
\question All of the following ions move across the neuronal membrane at different times \emph{EXCEPT}
\begin{choices}
\choice Na+
\choice K+
\choice Cl-
\correctchoice Organic anions (A-)
\end{choices}

\section{Bonus}

%41
\question During the \emph{falling} phase of the action potential, \fillin ions \fillin.
\begin{choices}
\correctchoice K+; flow out
\choice Na+; flow out
\choice K+; flow in
\choice Na+; flow in
\end{choices}

%42
\question The sympathetic nervous system is crucial for
\begin{choices}
\choice Sexual behavior
\choice Metabolic, physical support of neurons
\choice Sensory relay
\correctchoice Preparation for action
\choice Memory storage and retrieval
\end{choices}



%43
\question In a typical neuron near or slightly above its resting potential chloride (Cl-) ions would flow \fillin following the concentration gradient. This would move the neuron \fillin its firing threshold.
\begin{choices}
\correctchoice Inward; farther from
\choice Inward; closer to
\choice Outward; farther from
\choice Outward; closer to
\end{choices}

%44
\question A toxin found in Japanese pufferfish blocks voltage-gated Na+ channels. Applying such a toxin to neurons would have what effect?
\begin{choices}
\choice Slower falling phase of the action potential.
\choice Increasing the concentration of Na+ inside the cell.
\choice K+ ions would accelerate their flow to compensate.
\correctchoice Action potentials would be abolished.
\end{choices}

\end{questions}
\end{document}
